\documentclass[10pt, a4paper, onecolumn, oneside, titlepage, openany]{book}

\usepackage[T1]{fontenc}    %% TeX text extended - most European characters
\usepackage[utf8]{inputenc} %% UTF-8 support
\usepackage{menukeys}       %% keyboads keys, menu + includes '\definecolor'

\usepackage{hyperref}       %% URL and reference support
\hypersetup{
    colorlinks=true,
    linkcolor=blue,         %% \ref{}
    filecolor=magenta,      %% \href{run:./file.txt}{File.txt}
    urlcolor=cyan,          %% \href{http://www.overleaf.com}{Link}
    pdfpagemode=FullScreen,
    pdftitle={LFS}
}

\usepackage{titlesec}       %% chapter formatting (N. CHAPTER-NAME)
\renewcommand{\chaptername}{}
\titleformat{\chapter}[hang]{\normalfont\huge\bfseries}{\chaptertitlename\ \thechapter.}{1em}{}

\usepackage{fancyhdr}       %% decorative lines
\renewcommand{\headrulewidth}{2pt}
\renewcommand{\footrulewidth}{2pt}
\pagestyle{fancy}
\fancyhf{}
    \chead{\leftmark}
    \cfoot{\thepage}
\fancypagestyle{plain}{
\fancyhf{}
    \chead{\leftmark}
    \cfoot{\thepage}
}

%% table formatting
\usepackage[format=hang,font=small,labelfont=bf]{caption} %% bold table caption
\setlength{\arrayrulewidth}{0.5mm}                        %% border
\setlength{\tabcolsep}{18pt}                              %% space from border (X axis)
\renewcommand{\arraystretch}{1.5}                         %% space from border (Y axis)

%% colors
\definecolor{bg}{RGB}{240, 240, 240}
\definecolor{root}{RGB}{222, 0, 0}
\definecolor{user}{RGB}{0, 150, 0}
\definecolor{command}{RGB}{41, 182, 0}
\definecolor{block}{RGB}{255, 80, 0}
\definecolor{dir}{RGB}{0, 100, 200}
\definecolor{file}{RGB}{77, 187, 101}
\definecolor{comment}{RGB}{0, 182, 182}

\definecolor{keyword}{RGB}{204, 204, 0}

%% code formatting
\usepackage{fancyvrb}
\renewcommand{\FancyVerbFormatLine}[1]{\colorbox{bg}{#1}}

%% text file content formatting
\usepackage{listings}
%% Docs: https://texdoc.org/serve/listings.pdf/0 ; https://en.m.wikibooks.org/wiki/LaTeX/Source_Code_Listings ; % https://gordonlesti.com/custom-code-highlighting-in-latex/

\lstdefinelanguage{linux_file}{
    morekeywords = [1]{source,clear,set,umask,export},
    morekeywords = [2]{if,then,else,fi},
    sensitive = false,
    morecomment = [l]{\#},
    morestring = [b]",
    morestring = [b]',
}

\lstset{
    language = linux_file,
    basicstyle = \ttfamily,
    columns=fullflexible,
    keywordstyle = [1],%\color{command},
    keywordstyle = [2]\color{keyword},
    stringstyle = \color{block},
    commentstyle = \color{comment},
    keepspaces = true,
    numbers = left,
    firstnumber = 1,
    stepnumber = 1,
    numberstyle = \small\color{file},
    showspaces = false,
    showstringspaces = false,
    showtabs = false,
    breaklines = true,
    breakatwhitespace = false,
    breakindent=0pt,
    %postbreak=\space,
    %prebreak=\char92,
}

%%%%%%%%%%%%%%%%%%%%%%%%%%%%%%%%%%%%%%%%%%%%%%%%%%%%%%%%%%%%%%%%%%%%%%%%%%%%%%%%%%%%%%%%%%%%%%%%%%%%%%%%%%%%%%%%%

%% title page '\maketitle'
\title{\textbf{LFS}}
\author{AISK11}
\date{April 2022}

%% LaTeX content
\begin{document}
\maketitle
\tableofcontents

\chapter{System Info @WIP}
\begin{table}[!ht]
\centering
\begin{tabular}{|c|c|}
    \hline
    \textbf{Software} & \textbf{Value} \\
    \hline
    Distro name & Naomi\\
    Package manager & nmpm || naomi\\
    \hline
\end{tabular}
\caption{Distribution overview.}
\label{table:1}
\end{table}


%\section{System Hardware}
\begin{table}[!ht]
\centering
\begin{tabular}{|c|c|}
    \hline
    \textbf{Hardware} & \textbf{Value} \\
    \hline
    CPU architecture & amd64\\
    Multilib (32bit) support& yes\\
    CPU vendor & intel\\
    Motherboard & UEFI\\
    Hard Drive & NVMe\\
    WiFi & iwlwifi\\
    GPU & Intel + Nvidia\\
    \hline
\end{tabular}
\caption{System hardware overview.}
\label{table:2}
\end{table}

%\section{System Software}
\begin{table}[!ht]
\centering
\begin{tabular}{|c|c|}
    \hline
    \textbf{Software} & \textbf{Value} \\
    \hline
    RAID & -\\
    LVM & -\\
    Encryption & LUKS\\
    Root filesystem & btrfs\\
    SWAP & RAM + 2 GB (16 GB + 2 GB) [file]\\
    Bootloader & rEFInd\\
    Kernel & linux\\
    Initramfs & dracut\\
    Init & dinit/runit\\
    Shell & bash\\
    Virtualization & ?\\
    \hline
\end{tabular}
\caption{System software overview.}
\label{table:3}
\end{table}

%\section{Display Software}
\begin{table}[!ht]
\centering
\begin{tabular}{|c|c|}
    \hline
    \textbf{Software} & \textbf{Value} \\
    \hline
    Display server & Xorg\\
    Window manager & bspwm\\
    Desktop environment & -\\
    Terminal emulator & rxvt-unicode\\
    \hline
\end{tabular}
\caption{Display software overview.}
\label{table:4}
\end{table}


\chapter{Disk Preparation}
\section{Theory}
\begin{itemize}
    \item \textbf{Sector:} smallest unit size on disk. 512 or 4096 bytes (Advanced Format). 4096 Advanced Format disks have usually 512-byte conversion firmware.
    \item \textbf{Block:} allocation size the FS uses. Cannot be smaller than size of the sector. Can be group of sectors (4096b: 8 x 512b sectors).
    \begin{itemize}
        \item \textbf{512b =} good for lot of small files. More blocks = more metadata.
        \item \textbf{4096b =}  good for larger files, less metadata. Waste if there are mostly small files.
    \end{itemize}
\end{itemize}

\section{Reconnaissance}
\begin{itemize}
    \item \textbf{Find block devices:}
\begin{Verbatim}[commandchars=\\\{\}]
\textcolor{user}{user\$} \textcolor{command}{lsblk} [-ap | -apf]
\textcolor{root}{root#} \textcolor{command}{fdisk -l} [\textcolor{block}{/dev/sdX}]
\textcolor{root}{root#} \textcolor{command}{blkid}
\end{Verbatim}
\item \textbf{Raw block device info:}
    \begin{itemize}
        \item \textbf{Disk physical sector size:}
\begin{Verbatim}[commandchars=\\\{\}]
\textcolor{root}{root#} \textcolor{command}{blockdev} [-v] --getpbsz <\textcolor{block}{/dev/sdX[Y]}>
\end{Verbatim}
        \item \textbf{Disk logical sector size (usually 512):}
\begin{Verbatim}[commandchars=\\\{\}]
\textcolor{root}{root#} \textcolor{command}{blockdev} [-v] --getss <\textcolor{block}{/dev/sdX[Y]}>
\end{Verbatim}
        \item \textbf{Disk size in bytes:}
\begin{Verbatim}[commandchars=\\\{\}]
\textcolor{root}{root#} \textcolor{command}{blockdev} [-v] --getsize64 <\textcolor{block}{/dev/sdX[Y]}>
\end{Verbatim}
    \item \textbf{Check if disk is readonly (1 = ro, 0 = rw):}
\begin{Verbatim}[commandchars=\\\{\}]
\textcolor{root}{root#} \textcolor{command}{blockdev} [-v] --getro <\textcolor{block}{/dev/sdX[Y]}>
\end{Verbatim}
    \end{itemize}
    \item \textbf{List partitions:}
\begin{Verbatim}[commandchars=\\\{\}]
\textcolor{root}{root#} \textcolor{command}{parted} -s <\textcolor{block}{/dev/sdX}> (p)rint [free]
\end{Verbatim}
\end{itemize}

\section{Health Check}
\begin{itemize}
    \item \textbf{Check bad sectors:}
    \begin{enumerate}
        \item \textbf{Unmount FS!}
        \item \textbf{Check for bad blocks:}
\begin{Verbatim}[commandchars=\\\{\}]
\textcolor{root}{root#} \textcolor{command}{badblocks} [-b 4096] [-w [-t 0xaa]] [-v] [-s]
<\textcolor{block}{/dev/sdX[Y]}> | \textcolor{command}{tee} -a <\textcolor{file}{OUTPUT_FILE}>
\end{Verbatim}
    \end{enumerate}
\end{itemize}


\section{Partitioning}
\begin{enumerate}
    \item \textbf{Umount every FS from the disk that is going to be partitioned!}
    \item \textbf{Create partitions:}
    \begin{enumerate}
        \item \textbf{Enter cfdisk:}
\begin{Verbatim}[commandchars=\\\{\}]
\textcolor{root}{root#} \textcolor{command}{cfdisk} -z <\textcolor{block}{/dev/sdX}>
\end{Verbatim}
        \item \textbf{Create partition table:}
\begin{Verbatim}[commandchars=\\\{\}]
\textcolor{root}{cfdisk>} \textcolor{command}{gpt} (Enter)
\end{Verbatim}
        \item \textbf{Create EFI partition:}
\begin{Verbatim}[commandchars=\\\{\}]
\textcolor{root}{cfdisk>} \textcolor{command}{n}
\textcolor{root}{cfdisk>} \textcolor{command}{550MiB}
\textcolor{root}{cfdisk>} \textcolor{command}{t}
\textcolor{root}{cfdisk>} \textcolor{command}{EFI System}
\end{Verbatim}
        \item \textbf{Create LUKS partition:}
\begin{Verbatim}[commandchars=\\\{\}]
\textcolor{root}{cfdisk>} \textcolor{command}{n}
\textcolor{root}{cfdisk>} \textcolor{command}{} (Enter)
\end{Verbatim}
        \item \textbf{Write changes:}
\begin{Verbatim}[commandchars=\\\{\}]
\textcolor{root}{cfdisk>} \textcolor{command}{W}
\textcolor{root}{cfdisk>} \textcolor{command}{yes}
\end{Verbatim}
        \item \textbf{Quit cfdisk:}
\begin{Verbatim}[commandchars=\\\{\}]
\textcolor{root}{cfdisk>} \textcolor{command}{Q}
\end{Verbatim}
        \item \textbf{Name partitions:}
\begin{Verbatim}[commandchars=\\\{\}]
\textcolor{root}{root#} \textcolor{command}{parted} -s <\textcolor{block}{/dev/sdX}> name 1 ESP
\textcolor{root}{root#} \textcolor{command}{parted} -s <\textcolor{block}{/dev/sdX}> name 2 LUKS
\end{Verbatim}
    \end{enumerate}
    \item \textbf{Create filesystems:}
    \begin{enumerate}
        \item \textbf{ESP (VFAT32):}
\begin{Verbatim}[commandchars=\\\{\}]
\textcolor{root}{root#} \textcolor{command}{wipefs} -a <\textcolor{block}{/dev/sdX1}>
\textcolor{root}{root#} \textcolor{command}{mkfs.vfat} -F 32 <\textcolor{block}{/dev/sdX1}>
\textcolor{root}{root#} \textcolor{command}{fatlabel} <\textcolor{block}{/dev/sdX1}> ESP
\textcolor{root}{root#} \textcolor{command}{fatlabel} -i <\textcolor{block}{/dev/sdX1}> 00000001
\textcolor{root}{root#} \textcolor{command}{fsck.vfat} -v <\textcolor{block}{/dev/sdX1}> ; \textcolor{command}{echo} \$?
\end{Verbatim}
        \item \textbf{LUKS:}
\begin{Verbatim}[commandchars=\\\{\}]
\textcolor{root}{root#} \textcolor{command}{wipefs} -a <\textcolor{block}{/dev/sdX2}>
\textcolor{root}{root#} \textcolor{command}{cryptsetup} luksFormat -y --type luks2 <\textcolor{block}{/dev/sdX2}>
\textcolor{root}{>} YES
\textcolor{root}{>} <NEW_LUKS_PASSWORD>
\textcolor{root}{>} <NEW_LUKS_PASSWORD (VERIFY)>
\textcolor{root}{root#} \textcolor{command}{cryptsetup} config --label LUKS <\textcolor{block}{/dev/sdX2}>
\textcolor{root}{root#} \textcolor{command}{cryptsetup} luksUUID --uuid
00000000-0000-0000-0000-000000000002 <\textcolor{block}{/dev/sdX2}>
\textcolor{root}{>} YES
\textcolor{root}{root#} \textcolor{command}{cryptsetup} open --type luks2 <\textcolor{block}{/dev/sdX2}>
luks-root
\textcolor{root}{>} <PASSWORD>
\end{Verbatim}
        \item \textbf{Root FS (btrfs):}
\begin{Verbatim}[commandchars=\\\{\}]
\textcolor{root}{root#} \textcolor{command}{wipefs} -a \textcolor{block}{/dev/mapper/luks-root}
\textcolor{root}{root#} \textcolor{command}{mkfs.btrfs} \textcolor{block}{/dev/mapper/luks-root}
\textcolor{root}{root#} \textcolor{command}{btrfs} filesystem label \textcolor{block}{/dev/mapper/luks-root}
LUKS-ROOT
\textcolor{root}{root#} \textcolor{command}{btrfstune} -U 00000000-0000-0000-0000-000000000003
\textcolor{block}{/dev/mapper/luks-root}
\textcolor{root}{>} y
\textcolor{root}{root#} \textcolor{command}{btrfs} check -p \textcolor{block}{/dev/mapper/luks-root} ; \textcolor{command}{echo} \$?
\end{Verbatim}
    \end{enumerate}
    \item \textbf{Finish and clean:}
\begin{Verbatim}[commandchars=\\\{\}]
\textcolor{root}{root#} \textcolor{command}{cryptsetup} close \textcolor{block}{/dev/mapper/luks-root}
\end{Verbatim}
\end{enumerate}


\chapter{LFS Preparation}
\section{LFS User}
\begin{enumerate}
    \item \textbf{Create user:}
\begin{Verbatim}[commandchars=\\\{\}]
\textcolor{root}{root#} \textcolor{command}{useradd} -s \textcolor{file}{/bin/bash} -m -k \textcolor{block}{/dev/null} lfs
\textcolor{root}{root#} \textcolor{command}{passwd} lfs
\textcolor{root}{>} <NEW_USER_PASSWORD>
\textcolor{root}{>} <NEW_USER_PASSWORD (VERIFY)>
\end{Verbatim}
    \item \textbf{Edit user's bash config files:}
\newline File (\textbf{\textcolor{file}{/home/lfs/.bash\_profile}}):
\begin{lstlisting}[language=linux_file]
source ~/.bashrc
\end{lstlisting}
File (\textbf{\textcolor{file}{/home/lfs/.bashrc}}):
\begin{lstlisting}[language=linux_file]
umask 022
export LC_ALL="C"
export MAKEFLAGS="-j$(nproc)"
set +h
export PATH="/usr/bin"
if [[ ! -L /bin ]]; then export PATH="/bin:${PATH}"; fi
if [[ ! -L /usr/sbin ]]; then export PATH="${PATH}:/usr/sbin"; fi
if [[ ! -L /sbin ]]; then export PATH="${PATH}:/sbin"; fi
export PS1="\[\033[01;33m\]\u\[\033[01;30m\]@\[\033[01;31m\]\h \[\033[01;34m\]\w \[\033[01;30m\]\$ \[\033[01;37m\]"
\end{lstlisting}
    \item \textbf{Change file owner:}
\begin{Verbatim}[commandchars=\\\{\}]
\textcolor{root}{root#} \textcolor{command}{chown} -R lfs:lfs \textcolor{dir}{/home/lfs/}
\end{Verbatim}
    \item \textbf{Login as user:}
\begin{Verbatim}[commandchars=\\\{\}]
\textcolor{user}{user\$} \textcolor{command}{su} - lfs
\end{Verbatim}
\end{enumerate}

\section{Mounting}
\begin{enumerate}
    \item \textbf{Mount root fs:}
\begin{Verbatim}[commandchars=\\\{\}]
\textcolor{root}{root#} \textcolor{command}{mkdir} -p \textcolor{dir}{/mnt/lfs/}
\textcolor{root}{root#} \textcolor{command}{cryptsetup} open --type luks2 <\textcolor{block}{/dev/sdX2}>
luks-root
\textcolor{root}{>} <PASSWORD>
\textcolor{root}{root#} \textcolor{command}{mount} \textcolor{block}{/dev/mapper/luks-root} \textcolor{dir}{/mnt/lfs/}
\end{Verbatim}
    \item \textbf{Mount ESP partition:}
\begin{Verbatim}[commandchars=\\\{\}]
\textcolor{root}{root#} \textcolor{command}{mkdir} -p \textcolor{dir}{/mnt/lfs/boot/}
\textcolor{root}{root#} \textcolor{command}{mount} <\textcolor{block}{/dev/sdX1}> \textcolor{dir}{/mnt/lfs/boot/}
\end{Verbatim}
\end{enumerate}

\section{Directory Tree}
\begin{enumerate}
    \item \textbf{Rootless access:}
\begin{Verbatim}[commandchars=\\\{\}]
\textcolor{root}{root#} \textcolor{command}{chown} -R lfs:lfs \textcolor{dir}{/mnt/lfs/}
\end{Verbatim}
    \item \textbf{Binaries and libraries:}
\begin{Verbatim}[commandchars=\\\{\}]
\textcolor{user}{lfs\$} \textcolor{command}{mkdir} -p \textcolor{dir}{/mnt/lfs/usr/\{bin,lib,lib32\}/}
\textcolor{user}{lfs\$} \textcolor{command}{cd} \textcolor{dir}{/mnt/lfs/usr/}
\textcolor{user}{lfs\$} \textcolor{command}{ln} -sf \textcolor{dir}{bin/} \textcolor{file}{sbin}
\textcolor{user}{lfs\$} \textcolor{command}{ln} -sf \textcolor{dir}{lib/} \textcolor{file}{lib64}
\textcolor{user}{user\$} \textcolor{command}{cd} \textcolor{dir}{/mnt/lfs/}
\textcolor{user}{lfs\$} \textcolor{command}{ln} -sf \textcolor{dir}{usr/bin/} \textcolor{file}{bin}
\textcolor{user}{lfs\$} \textcolor{command}{ln} -sf \textcolor{dir}{usr/sbin/} \textcolor{file}{sbin}
\textcolor{user}{lfs\$} \textcolor{command}{ln} -sf \textcolor{dir}{usr/lib/} \textcolor{file}{lib}
\textcolor{user}{lfs\$} \textcolor{command}{ln} -sf \textcolor{dir}{usr/lib64/} \textcolor{file}{lib64}
\textcolor{user}{lfs\$} \textcolor{command}{ln} -sf \textcolor{dir}{usr/lib32/} \textcolor{file}{lib32}
\end{Verbatim}
    \item \textbf{Other directories:}
\begin{Verbatim}[commandchars=\\\{\}]
\textcolor{user}{lfs\$} \textcolor{command}{mkdir} -p \textcolor{dir}{/mnt/lfs/\{etc,var\}/}
\end{Verbatim}
    \item \textbf{Source code and compile tools:}
\begin{Verbatim}[commandchars=\\\{\}]
\textcolor{user}{lfs\$} \textcolor{command}{mkdir} -p \textcolor{dir}{/mnt/lfs/\{lfs_sources,lfs_tools\}/}
\end{Verbatim}
\end{enumerate}


\chapter{Cross Toolchain}
\section{Binutils}
\textit{Linker, assembler and tools for handling object files.}
\begin{enumerate}
    \item \textbf{Download binutils:}
    \begin{itemize}
        \item \textbf{Webpage:} \url{https://www.gnu.org/software/binutils/}
        \item \textbf{Download}
        \begin{itemize}
            \item \textbf{Repository:} \url{https://ftp.gnu.org/gnu/binutils/}
            \item \textbf{Mirror bouncer:} \url{https://ftpmirror.gnu.org/binutils/}
        \end{itemize}
    \end{itemize}
    \item \textbf{Verify binutils:}
    \begin{enumerate}
        \item \textbf{Download GPG keyring:}
\begin{Verbatim}[commandchars=\\\{\}]
\textcolor{user}{lfs\$} \textcolor{command}{wget} \underline{https://ftp.gnu.org/gnu/gnu-keyring.gpg}
\end{Verbatim}
        \item \textbf{Verify signature with keyring:}
\begin{Verbatim}[commandchars=\\\{\}]
\textcolor{user}{lfs\$} \textcolor{command}{gpg} --verify --keyring
<\textcolor{file}{gnu-keyring.gpg}> <\textcolor{file}{SIG-FILE.sig}>
\end{Verbatim}
        \item \textbf{Remove verifications:}
\begin{Verbatim}[commandchars=\\\{\}]
\textcolor{user}{lfs\$} \textcolor{command}{rm} <\textcolor{file}{SIG-FILE.sig}]> [\textcolor{file}{gnu-keyring.gpg}]
\end{Verbatim}
    \end{enumerate}
    \item \textbf{Export binutils:}
\begin{Verbatim}[commandchars=\\\{\}]
\textcolor{user}{lfs\$} \textcolor{command}{mv} <\textcolor{file}{binutils-X.YY.tar.xz}> \textcolor{dir}{/mnt/lfs/lfs_sources/}
\textcolor{user}{lfs\$} \textcolor{command}{cd} \textcolor{dir}{/mnt/lfs/lfs_sources/}
\textcolor{user}{lfs\$} \textcolor{command}{tar} xvJf <\textcolor{file}{binutils-X.YY.tar.xz}>
\textcolor{user}{lfs\$} \textcolor{command}{rm} <\textcolor{file}{binutils-X.YY.tar.xz}>
\end{Verbatim}
    \item \textbf{Compile binutils:}
\begin{Verbatim}[commandchars=\\\{\}]
\textcolor{user}{lfs\$} \textcolor{command}{mkdir} <\textcolor{dir}{/mnt/lfs/lfs_sources/binutils-X.YY/build/}>
\textcolor{user}{lfs\$} \textcolor{command}{cd} <\textcolor{dir}{/mnt/lfs/lfs_sources/binutils-X.YY/build/}>
\textcolor{user}{lfs\$} \textcolor{command}{../configure} --prefix=\textcolor{dir}{/mnt/lfs/lfs_tools/}
--with-sysroot=\textcolor{dir}{/mnt/lfs/} --target=\$(uname -m)-lfs-linux-gnu
--disable-nls --disable-werror
\textcolor{user}{lfs\$} \textcolor{command}{time} \textcolor{command}{make}
\textcolor{user}{lfs\$} \textcolor{command}{make} install
\end{Verbatim}
    \item \textbf{Remove source:}
\begin{Verbatim}[commandchars=\\\{\}]
\textcolor{user}{lfs\$} \textcolor{command}{cd} \textcolor{dir}{/mnt/lfs/lfs_sources/}
\textcolor{user}{lfs\$} \textcolor{command}{rm} -rf <\textcolor{dir}{binutils-X.YY/}>
\end{Verbatim}
\end{enumerate}

\section{GCC}
\textit{C and C++ compilers.}
\subsection{MPFR}
\begin{enumerate}
    \item \textbf{Download MPFR:}
    \begin{itemize}
        \item \textbf{Webpage:} \url{https://www.mpfr.org/}
        \item \textbf{Download}
        \begin{itemize}
            \item \textbf{Lates release:} \url{https://www.mpfr.org/mpfr-current/}
            \item \textbf{Previous releases:} \url{https://www.mpfr.org/history.html}
        \end{itemize}
    \end{itemize}
    \item \textbf{Verify MPFR:}
    \begin{enumerate}
        \item \textbf{Download MPFR author public PGP key:}
\begin{Verbatim}[commandchars=\\\{\}]
\textcolor{user}{lfs\$} \textcolor{command}{wget} -O <\textcolor{file}{mpfr_key.asc}> \underline{https://www.vinc17.net/key.asc}
\end{Verbatim}



    \end{enumerate}
\end{enumerate}




\chapter{Skipped}
\begin{itemize}
    \item \textbf{Download of packages:} (\url{https://www.linuxfromscratch.org/lfs/view/stable/chapter03/packages.html})
    \item \textbf{Filesystem Hierarchy}
    \item \textbf{Possible wrong build binaries}
    \item \textbf{LFS user bash profile} (\url{https://www.linuxfromscratch.org/lfs/view/stable/chapter04/settingenvironment.html})
\end{itemize}


\chapter{Packages}
\section{Core}
\begin{itemize}
    \item \textbf{binutils:}
    \begin{itemize}
        \item \textbf{Webpage:} \url{https://www.gnu.org/software/binutils/}
    \end{itemize}
\end{itemize}


\chapter{Cheat Sheets}
\section{LUKS}
\label{cheatsheet_luks}
\begin{itemize}
    \item \textbf{Open:}
\begin{Verbatim}[commandchars=\\\{\}]
\textcolor{root}{root#} \textcolor{command}{cryptsetup} open --type luks <\textcolor{block}{/dev/sdX2}> <luks>
> <PASSWORD>
\end{Verbatim}
    \item \textbf{Close:}
\begin{Verbatim}[commandchars=\\\{\}]
\textcolor{root}{root#} \textcolor{command}{cryptsetup} close <luks>
\end{Verbatim}
    \item \textbf{Header:}
    \begin{enumerate}
        \item \textbf{See LUKS header:}
\begin{Verbatim}[commandchars=\\\{\}]
\textcolor{root}{root#} \textcolor{command}{cryptsetup} luksDump <\textcolor{block}{/dev/sdX2}>
\end{Verbatim}
        \item \textbf{Make LUKS header backup:}
\begin{Verbatim}[commandchars=\\\{\}]
\textcolor{root}{root#} \textcolor{command}{cryptsetup} luksHeaderBackup <\textcolor{block}{/dev/sdX2}>
--header-backup-file <\textcolor{file}{FILE}>
\end{Verbatim}
        \item \textbf{Destroy LUKS header:}
\begin{Verbatim}[commandchars=\\\{\}]
\textcolor{root}{root#} \textcolor{command}{cryptsetup} luksErase <\textcolor{block}{/dev/sdX2}>
> YES
\end{Verbatim}
        \item \textbf{Restore LUKS header:}
\begin{Verbatim}[commandchars=\\\{\}]
\textcolor{root}{root#} \textcolor{command}{cryptsetup} luksHeaderRestore <\textcolor{block}{/dev/sdX2}>
--header-backup-file <\textcolor{file}{FILE}>
> YES
\end{Verbatim}
    \end{enumerate}
    \item \textbf{Passwords:}
    \begin{itemize}
        \item \textbf{Change password:}
\begin{Verbatim}[commandchars=\\\{\}]
\textcolor{root}{root#} \textcolor{command}{cryptsetup} luksChangeKey <\textcolor{block}{/dev/sdX2}>
> <OLD_PASSWORD>
> <NEW_PASSWORD>
> <NEW_PASSWORD (VERIFY)>
\end{Verbatim}
    \end{itemize}
\end{itemize}


\chapter{Reference}
Current progress:
\newline \url{https://www.linuxfromscratch.org/lfs/view/stable/chapter05/binutils-pass1.html}
\newline \url{https://www.youtube.com/watch?v=6Mw4l0khpCU}
\section{Linux From Scratch}
\begin{itemize}
    \item \textbf{Homepage:} \url{https://www.linuxfromscratch.org/}
    \item \textbf{LFS:} \url{https://www.linuxfromscratch.org/lfs/view/stable/}
    \item \textbf{LFS multilib fork:} \url{https://www.linuxfromscratch.org/~thomas/multilib/index.html}
    \item \textbf{BFLS:} \url{https://www.linuxfromscratch.org/blfs/downloads/stable/BLFS-BOOK-11.1-nochunks.html}
\end{itemize}

\section{Disks}
\begin{itemize}
    \item \textbf{FS comparison:} \url{https://en.wikipedia.org/wiki/Comparison_of_file_systems}
\end{itemize}

\section{Linux Filesystem Hierarchy Standard}
\begin{itemize}
    \item \textbf{Linux FHS:} \url{https://www.pathname.com/fhs/pub/fhs-2.3.html}
\end{itemize}

\section{Packages}
\begin{itemize}
    \item \textbf{Package info:} \url{https://www.linuxfromscratch.org/lfs/view/stable/prologue/package-choices.html}
    \item \textbf{Package URLs:} \url{https://www.linuxfromscratch.org/lfs/view/stable/chapter03/packages.html}
\end{itemize}


\chapter{GPG (OpenPGP)}
\section{Legend}
\begin{itemize}
    \item \textbf{Key Fingerprint:} 160-bit key identifier (0000 1111 2222 3333 4444 BBBB CCCC DDDD EEEE FFFF).
    \item \textbf{Long Key ID:} last 16 chars of fingerprint (CCCC DDDD EEEE FFFF).
    \item \textbf{Short Key ID:} last 8 chars of fingerprint (EEEE FFFF).
    \item \textbf{User ID (UID):} "NAME SURNAME <EMAIL@ADDRESS>"
\end{itemize}

\begin{table}[!ht]
\centering
\begin{tabular}{|c|c|}
    \hline
    \textbf{Shortcut} & \textbf{Key} \\
    \hline
    pub & PUBlic key\\
    sub & public SUBkey\\
    sec & SECret key\\
    ssb & Secret SuBkey\\
    uid & User ID\\
    \hline
\end{tabular}
\caption{GPG keys.}
\label{table:gpg_keys}
\end{table}

\begin{table}[!ht]
\centering
\begin{tabular}{|c|c|c|}
    \hline
    \textbf{Constant} & \textbf{X} & \textbf{Explanation} \\
    \hline
    PUBKEY\_USAGE\_SIG & S & key is good for signing \\
    PUBKEY\_USAGE\_CERT & C & key is good for certifying other signatures\\
    PUBKEY\_USAGE\_ENC & E & key is good for encryption\\
    PUBKEY\_USAGE\_AUTH & A & key is good for authentication\\
    \hline
\end{tabular}
\caption{GPG usage flags.}
\label{table:gpg_usage_flags}
\end{table}

\begin{table}[!ht]
\centering
\begin{tabular}{|c|c|c|}
    \hline
    \textbf{N} & \textbf{Trust Level} & \textbf{Explanation} \\
    \hline
    6 & Ultimate & my own keys\\
    5 & Full & trust to sign other keys\\
    4 & Marginal & complex\\
    3 & & \\
    2 & & \\
    \hline
\end{tabular}
\caption{GPG trust levels.}
\label{table:gpg_trust_levels}
\end{table}

\section{Installation}
\begin{itemize}
    \item \textbf{LFS}
\newline \url{https://www.gnupg.org/download/index.html}
\end{itemize}

\section{Configuration}
\begin{enumerate}
    \item \textbf{Change GPG home dir:}
\newline Default: \textbf{\textcolor{dir}{\texttildelow/.gnupg/}}
\begin{Verbatim}[commandchars=\\\{\}]
\textcolor{user}{user\$} GNUPGHOME='<\textcolor{dir}{DIRECTORY/}>'
\end{Verbatim}
\end{enumerate}

\section{Create Key Pair}
\begin{enumerate}
    \item \textbf{Generate keys:}
\begin{Verbatim}[commandchars=\\\{\}]
\textcolor{user}{user\$} gpg --full-gen-key
\textcolor{user}{>} (1) RSA and RSA (default)
\textcolor{user}{>} 4096
\textcolor{user}{>} 0
\textcolor{user}{>} y
\textcolor{user}{>} <NAME> [SURNAME]
\textcolor{user}{>} <EMAIL>
\textcolor{user}{>} [COMMENT]
\textcolor{user}{>} O
\textcolor{user}{>} <PASSPHRASE>
\textcolor{user}{>} <PASSPHRASE (VERIFY)>
\end{Verbatim}
\end{enumerate}

\section{List Keys}
\begin{itemize}
    \item \textbf{List Public keys:}
\begin{Verbatim}[commandchars=\\\{\}]
\textcolor{user}{user\$} \textcolor{command}{gpg} -k [--keyid-format long]
\end{Verbatim}
    \item \textbf{List Secret/Private keys:}
\begin{Verbatim}[commandchars=\\\{\}]
\textcolor{user}{user\$} \textcolor{command}{gpg} -K [--keyid-format long]
\end{Verbatim}
\end{itemize}

\section{References}
\begin{itemize}
    \item \url{https://gnupg.org/}
    \item \url{https://wiki.gentoo.org/wiki/GnuPG}
    \item \url{https://wiki.archlinux.org/title/GnuPG}
    \item \url{https://www.void.gr/kargig/blog/2013/12/02/creating-a-new-gpg-key-with-subkeys/}
    \item \url{https://wiki.debian.org/Subkeys}
    \item \url{https://guides.library.illinois.edu/data_encryption/gpgcheatsheet}
    \item \url{https://davesteele.github.io/gpg/2014/09/20/anatomy-of-a-gpg-key/}
\end{itemize}

\end{document}
