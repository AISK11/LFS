\documentclass[10pt, a4paper, onecolumn, oneside, titlepage, openany]{book}

\usepackage[T1]{fontenc}    %% TeX text extended - most European characters
\usepackage[utf8]{inputenc} %% UTF-8 support
\usepackage{menukeys}       %% keyboads keys, menu + includes '\definecolor'

\usepackage{hyperref}       %% URL and reference support
\hypersetup{
    colorlinks=true,
    linkcolor=blue,         %% \ref{}
    filecolor=magenta,      %% \href{run:./file.txt}{File.txt}
    urlcolor=cyan,          %% \href{http://www.overleaf.com}{Link}
    pdfpagemode=FullScreen,
    pdftitle={LFS}
}

\usepackage{titlesec}       %% chapter formatting (N. CHAPTER-NAME)
\renewcommand{\chaptername}{}
\titleformat{\chapter}[hang]{\normalfont\huge\bfseries}{\chaptertitlename\ \thechapter.}{1em}{}

\usepackage{fancyhdr}       %% decorative lines
\renewcommand{\headrulewidth}{2pt}
\renewcommand{\footrulewidth}{2pt}
\pagestyle{fancy}
\fancyhf{}
    \chead{\leftmark}
    \cfoot{\thepage}
\fancypagestyle{plain}{
\fancyhf{}
    \chead{\leftmark}
    \cfoot{\thepage}
}

%% table formatting
\usepackage[format=hang,font=small,labelfont=bf]{caption} %% bold table caption
\setlength{\arrayrulewidth}{0.5mm}                        %% border
\setlength{\tabcolsep}{18pt}                              %% space from border (X axis)
\renewcommand{\arraystretch}{1.5}                         %% space from border (Y axis)

%% colors
\definecolor{bg}{RGB}{240, 240, 240}
\definecolor{root}{RGB}{222, 0, 0}
\definecolor{user}{RGB}{0, 150, 0}
\definecolor{command}{RGB}{41, 182, 0}
\definecolor{block}{RGB}{255, 80, 0}
\definecolor{dir}{RGB}{0, 100, 200}
\definecolor{file}{RGB}{77, 187, 101}
\definecolor{comment}{RGB}{0, 182, 182}

%% code formatting
\usepackage{fancyvrb}
\renewcommand{\FancyVerbFormatLine}[1]{\colorbox{bg}{#1}}

%%%%%%%%%%%%%%%%%%%%%%%%%%%%%%%%%%%%%%%%%%%%%%%%%%%%%%%%%%%%%%%%%%%%%%%%%%%%%%%%%%%%%%%%%%%%%%%%%%%%%%%%%%%%%%%%%
%% code formatting - TEST!!!
%% Docs: https://texdoc.org/serve/listings.pdf/0 ; https://en.m.wikibooks.org/wiki/LaTeX/Source_Code_Listings ; % https://gordonlesti.com/custom-code-highlighting-in-latex/
\usepackage{listings}
%\definecolor{black}{rgb}{0, 0, 0}
%\definecolor{bg}{rgb}{0.98, 0.98, 0.82}
%\definecolor{user}{rgb}{0.31, 0.49, 0.16}
%\definecolor{command}{rgb}{0.0, 1.0, 0.0}
%\definecolor{comment}{rgb}{0.5, 0.5, 0.5}
%\definecolor{string}{rgb}{0.89, 0.35, 0.13}

%\lstdefinelanguage{linux}{
%    morekeywords = [1]{root}
%    morekeywords = [2]{user}
%    morekeywords = [3]{pwd,cd,ls,mv,cp,rm,mkdir,rmdir,dd,ln,
%        less,zless,bzless,xzless,more,zmore,bzmore,xzmore,cat,zcat,bzcat,xzcat,grep,zgrep,bzgrep,xzgrep,nl,wc,echo,file,stat,zip,unzip,tar,gzip,gunzip,xz,bzip2,bunzip2,help,man,info,tldr
%        whoami,id,groups,passwd,chsh,su,sudo,doas,chown,chmod,chgrp,useradd,usermod,userdel,groupadd,groupmod,groupdel,
%        df,du,mount,umount,chroot,lsblk,blkid,blockdev,badblocks,parted,gdisk,fdisk,cfdisk,cryptsetup,mkfs.fat,fatlabel,mkfs.btrfs,btrfs,truncate,chattr,fallocate,mkswap,swapon,swapoff,mkfs.ext4,
%        dmesg,lsmod,modprobe,lspci,lsusb,inxi,usbguard},
%    keywordstyle = [1]\color{command},
%    keywordstyle = [2]\color{user},
%    keywordstyle = [3]\color{command},
%    sensitive = true,
%    morecomment = [l]{\#},
%    %morecomment = [s]{/*}{*/},
%    %morecomment = [s]{/**}{*/},
%    commentstyle = \color{comment},
%    morestring = [b]",
%    morestring = [b]',
%    stringstyle = \color{string},
%    backgroundcolor = \color{bg},
%}

%\lstset{
%  basicstyle = \footnotesize,        % the size of the fonts that are used for the code
%  breakatwhitespace = false,         % sets if automatic breaks should only happen at whitespace
%  breaklines = true,                 % sets automatic line breaking
%  captionpos = b,                    % sets the caption-position to bottom
%  escapeinside = {\%*}{*)},          % if you want to add LaTeX within your code
%  extendedchars = true,              % lets you use non-ASCII characters; for 8-bits encodings only, does not work with UTF-8
%  firstnumber = 0,                % start line enumeration with line 1000
%  frame = single,	                   % adds a frame around the code
%  keepspaces = true,                 % keeps spaces in text, useful for keeping indentation of code (possibly needs columns=flexible)
%  language = Octave,                 % the language of the code
%  morekeywords = {*,...},            % if you want to add more keywords to the set
%  numbers = left,                    % where to put the line-numbers; possible values are (none, left, right)
%  numbersep = 5pt,                   % how far the line-numbers are from the code
%  numberstyle = \tiny\color{black}, % the style that is used for the line-numbers
%  rulecolor = \color{black},         % if not set, the frame-color may be changed on line-breaks within not-black text (e.g. comments (green here))
%  showspaces = false,                % show spaces everywhere adding particular underscores; it overrides 'showstringspaces'
%  showstringspaces = false,          % underline spaces within strings only
%  showtabs = false,                  % show tabs within strings adding particular underscores
%  stepnumber = 1,                    % the step between two line-numbers. If it's 1, each line will be numbered
%  tabsize = 4,	                   % sets default tabsize to 2 spaces
%  basicstyle={\small},
%  identifierstyle={\small},
%  commentstyle={\small\itshape},
%  keywordstyle={\small\bfseries},
%  ndkeywordstyle={\small},
%  stringstyle={\small\ttfamily},
%  frame={tb},
%  breaklines=true,
%  columns=[l]{fullflexible},
%  numbers=left,
%  xrightmargin=0em,
%  xleftmargin=3em,
%  numberstyle={\scriptsize},
%  stepnumber=1,
%  numbersep=1em,
%  lineskip=-0.5ex,
%}
%%%%%%%%%%%%%%%%%%%%%%%%%%%%%%%%%%%%%%%%%%%%%%%%%%%%%%%%%%%%%%%%%%%%%%%%%%%%%%%%%%%%%%%%%%%%%%%%%%%%%%%%%%%%%%%%%

%% title page '\maketitle'
\title{\textbf{LFS}}
\author{AISK11}
\date{April 2022}

%% LaTeX content
\begin{document}
\maketitle
\tableofcontents

\chapter{Disk Preparation}
\section{Theory}
\begin{itemize}
    \item \textbf{Sector:} smallest unit size on disk. 512 or 4096 bytes (Advanced Format). 4096 Advanced Format disks have usually 512-byte conversion firmware.
    \item \textbf{Block:} allocation size the FS uses. Cannot be smaller than size of the sector. Can be group of sectors (4096b: 8 x 512b sectors).
    \begin{itemize}
        \item \textbf{512b =} good for lot of small files. More blocks = more metadata.
        \item \textbf{4096b =}  good for larger files, less metadata. Waste if there are mostly small files.
    \end{itemize}
\end{itemize}

\section{Reconnaissance}
\begin{itemize}
    \item \textbf{Find block devices:}
\begin{Verbatim}[commandchars=\\\{\}]
\textcolor{user}{user\$} \textcolor{command}{lsblk} [-ap | -apf]
\textcolor{root}{root#} \textcolor{command}{fdisk -l} [\textcolor{block}{/dev/sdX}]
\textcolor{root}{root#} \textcolor{command}{blkid}
\end{Verbatim}
\item \textbf{Raw block device info:}
    \begin{itemize}
        \item \textbf{Disk physical sector size:}
\begin{Verbatim}[commandchars=\\\{\}]
\textcolor{root}{root#} \textcolor{command}{blockdev} [-v] --getpbsz <\textcolor{block}{/dev/sdX[Y]}>
\end{Verbatim}
        \item \textbf{Disk logical sector size (usually 512):}
\begin{Verbatim}[commandchars=\\\{\}]
\textcolor{root}{root#} \textcolor{command}{blockdev} [-v] --getss <\textcolor{block}{/dev/sdX[Y]}>
\end{Verbatim}
        \item \textbf{Disk size in bytes:}
\begin{Verbatim}[commandchars=\\\{\}]
\textcolor{root}{root#} \textcolor{command}{blockdev} [-v] --getsize64 <\textcolor{block}{/dev/sdX[Y]}>
\end{Verbatim}
    \item \textbf{Check if disk is readonly (1 = ro, 0 = rw):}
\begin{Verbatim}[commandchars=\\\{\}]
\textcolor{root}{root#} \textcolor{command}{blockdev} [-v] --getro <\textcolor{block}{/dev/sdX[Y]}>
\end{Verbatim}
    \end{itemize}
    \item \textbf{List partitions:}
\begin{Verbatim}[commandchars=\\\{\}]
\textcolor{root}{root#} \textcolor{command}{parted} -s <\textcolor{block}{/dev/sdX}> (p)rint [free]
\end{Verbatim}
\end{itemize}

\section{Health Check}
\begin{itemize}
    \item \textbf{Check bad sectors:}
    \begin{enumerate}
        \item \textbf{Unmount FS!}
        \item \textbf{Check for bad blocks:}
\begin{Verbatim}[commandchars=\\\{\}]
\textcolor{root}{root#} \textcolor{command}{badblocks} [-b 4096] [-w [-t 0xaa]] [-v] [-s]
<\textcolor{block}{/dev/sdX[Y]}> | \textcolor{command}{tee} -a <\textcolor{file}{OUTPUT_FILE}>
\end{Verbatim}
    \end{enumerate}
\end{itemize}


\section{Partitioning}
\begin{enumerate}
    \item \textbf{Umount every FS from the disk that is going to be partitioned!}
    \item \textbf{Create partitions:}
    \begin{enumerate}
        \item \textbf{Enter cfdisk:}
\begin{Verbatim}[commandchars=\\\{\}]
\textcolor{root}{root#} \textcolor{command}{cfdisk} -z <\textcolor{block}{/dev/sdX}>
\end{Verbatim}
        \item \textbf{Create partition table:}
\begin{Verbatim}[commandchars=\\\{\}]
\textcolor{root}{cfdisk>} \textcolor{command}{gpt} (Enter)
\end{Verbatim}
        \item \textbf{Create EFI partition:}
\begin{Verbatim}[commandchars=\\\{\}]
\textcolor{root}{cfdisk>} \textcolor{command}{n}
\textcolor{root}{cfdisk>} \textcolor{command}{550MiB}
\textcolor{root}{cfdisk>} \textcolor{command}{t}
\textcolor{root}{cfdisk>} \textcolor{command}{EFI System}
\end{Verbatim}
        \item \textbf{Create LUKS partition:}
\begin{Verbatim}[commandchars=\\\{\}]
\textcolor{root}{cfdisk>} \textcolor{command}{n}
\textcolor{root}{cfdisk>} \textcolor{command}{} (Enter)
\end{Verbatim}
        \item \textbf{Write changes:}
\begin{Verbatim}[commandchars=\\\{\}]
\textcolor{root}{cfdisk>} \textcolor{command}{W}
\textcolor{root}{cfdisk>} \textcolor{command}{yes}
\end{Verbatim}
        \item \textbf{Quit cfdisk:}
\begin{Verbatim}[commandchars=\\\{\}]
\textcolor{root}{cfdisk>} \textcolor{command}{Q}
\end{Verbatim}
        \item \textbf{Name partitions:}
\begin{Verbatim}[commandchars=\\\{\}]
\textcolor{root}{root#} \textcolor{command}{parted} -s <\textcolor{block}{/dev/sdX}> name 1 ESP
\textcolor{root}{root#} \textcolor{command}{parted} -s <\textcolor{block}{/dev/sdX}> name 2 LUKS
\end{Verbatim}
    \end{enumerate}
    \item \textbf{Create filesystems:}
    \begin{enumerate}
        \item \textbf{ESP (VFAT32):}
\begin{Verbatim}[commandchars=\\\{\}]
\textcolor{root}{root#} \textcolor{command}{wipefs} -a <\textcolor{block}{/dev/sdX1}>
\textcolor{root}{root#} \textcolor{command}{mkfs.vfat} -F 32 <\textcolor{block}{/dev/sdX1}>
\textcolor{root}{root#} \textcolor{command}{fatlabel} <\textcolor{block}{/dev/sdX1}> ESP
\textcolor{root}{root#} \textcolor{command}{fatlabel} -i <\textcolor{block}{/dev/sdX1}> 00000001
\textcolor{root}{root#} \textcolor{command}{fsck.vfat} -v <\textcolor{block}{/dev/sdX1}> ; \textcolor{command}{echo} \$?
\end{Verbatim}
        \item \textbf{LUKS:}
\begin{Verbatim}[commandchars=\\\{\}]
\textcolor{root}{root#} \textcolor{command}{wipefs} -a <\textcolor{block}{/dev/sdX2}>
\textcolor{root}{root#} \textcolor{command}{cryptsetup} luksFormat --type luks2 --label LUKS
<\textcolor{block}{/dev/sdX2}>
\textcolor{root}{>} YES
\textcolor{root}{>} <NEW_LUKS_PASSWORD>
\textcolor{root}{>} <NEW_LUKS_PASSWORD (VERIFY)>
\textcolor{root}{root#} \textcolor{command}{cryptsetup} luksUUID --uuid
00000000-0000-0000-0000-000000000002 <\textcolor{block}{/dev/sdX2}>
\textcolor{root}{>} YES
\textcolor{root}{root#} \textcolor{command}{cryptsetup} open --type luks2 <\textcolor{block}{/dev/sdX2}>
luks-root
\textcolor{root}{>} <PASSWORD>
\end{Verbatim}
        \item \textbf{Root FS (btrfs):}
\begin{Verbatim}[commandchars=\\\{\}]
\textcolor{root}{root#} \textcolor{command}{wipefs} -a \textcolor{block}{/dev/mapper/luks-root}
\textcolor{root}{root#} \textcolor{command}{mkfs.btrfs} \textcolor{block}{/dev/mapper/luks-root}
\textcolor{root}{root#} \textcolor{command}{btrfs} filesystem label \textcolor{block}{/dev/mapper/luks-root}
LUKS-ROOT
\textcolor{root}{root#} \textcolor{command}{btrfstune} -U 00000000-0000-0000-0000-000000000003
\textcolor{block}{/dev/mapper/luks-root}
\textcolor{root}{>} y
\textcolor{root}{root#} \textcolor{command}{btrfs} check -p \textcolor{block}{/dev/mapper/luks-root} ; \textcolor{command}{echo} \$?
\end{Verbatim}
    \end{enumerate}
\end{enumerate}


\chapter{LFS Installation}
\section{Mounting}
\begin{enumerate}
    \item \textbf{Mount root fs:}
\begin{Verbatim}[commandchars=\\\{\}]
\textcolor{root}{root#} \textcolor{command}{mkdir} -p \textcolor{dir}{/mnt/lfs/}
\textcolor{root}{root#} \textcolor{command}{mount} \textcolor{block}{/dev/mapper/luks-root} \textcolor{dir}{/mnt/lfs/}
\end{Verbatim}
    \item \textbf{Mount ESP partition:}
\begin{Verbatim}[commandchars=\\\{\}]
\textcolor{root}{root#} \textcolor{command}{mkdir} -p \textcolor{dir}{/mnt/lfs/boot/}
\textcolor{root}{root#} \textcolor{command}{mount} <\textcolor{block}{/dev/sdX1}> \textcolor{dir}{/mnt/lfs/boot/}
\end{Verbatim}
\end{enumerate}

\section{Basic Packages}
\begin{enumerate}
    \item \textbf{Create source code dir:}
\begin{Verbatim}[commandchars=\\\{\}]
\textcolor{root}{root#} \textcolor{command}{mkdir} -p \textcolor{dir}{/mnt/lfs/sources/}
\textcolor{root}{root#} \textcolor{command}{chmod} 1777 \textcolor{dir}{/mnt/lfs/sources/}
\end{Verbatim}
\end{enumerate}


\chapter{Cheat Sheets}
\section{LUKS}
\label{cheatsheet_luks}
\begin{itemize}
    \item \textbf{Open:}
\begin{Verbatim}[commandchars=\\\{\}]
\textcolor{root}{root#} \textcolor{command}{cryptsetup} open --type luks <\textcolor{block}{/dev/sdX2}> <luks>
> <PASSWORD>
\end{Verbatim}
    \item \textbf{Close:}
\begin{Verbatim}[commandchars=\\\{\}]
\textcolor{root}{root#} \textcolor{command}{cryptsetup} close <luks>
\end{Verbatim}
    \item \textbf{Header:}
    \begin{enumerate}
        \item \textbf{See LUKS header:}
\begin{Verbatim}[commandchars=\\\{\}]
\textcolor{root}{root#} \textcolor{command}{cryptsetup} luksDump <\textcolor{block}{/dev/sdX2}>
\end{Verbatim}
        \item \textbf{Make LUKS header backup:}
\begin{Verbatim}[commandchars=\\\{\}]
\textcolor{root}{root#} \textcolor{command}{cryptsetup} luksHeaderBackup <\textcolor{block}{/dev/sdX2}>
--header-backup-file <\textcolor{file}{FILE}>
\end{Verbatim}
        \item \textbf{Destroy LUKS header:}
\begin{Verbatim}[commandchars=\\\{\}]
\textcolor{root}{root#} \textcolor{command}{cryptsetup} luksErase <\textcolor{block}{/dev/sdX2}>
> YES
\end{Verbatim}
        \item \textbf{Restore LUKS header:}
\begin{Verbatim}[commandchars=\\\{\}]
\textcolor{root}{root#} \textcolor{command}{cryptsetup} luksHeaderRestore <\textcolor{block}{/dev/sdX2}>
--header-backup-file <\textcolor{file}{FILE}>
> YES
\end{Verbatim}
    \end{enumerate}
    \item \textbf{Passwords:}
    \begin{itemize}
        \item \textbf{Change password:}
\begin{Verbatim}[commandchars=\\\{\}]
\textcolor{root}{root#} \textcolor{command}{cryptsetup} luksChangeKey <\textcolor{block}{/dev/sdX2}>
> <OLD_PASSWORD>
> <NEW_PASSWORD>
> <NEW_PASSWORD (VERIFY)>
\end{Verbatim}
    \end{itemize}
\end{itemize}


\chapter{Reference}
Current progress:
\newline \url{https://www.linuxfromscratch.org/lfs/view/stable/chapter03/packages.html}
\newline \url{https://www.youtube.com/watch?v=iKeVxGq4QLw}
\section{Linux From Scratch}
\begin{itemize}
    \item \textbf{Homepage:} \url{https://www.linuxfromscratch.org/}
    \item \textbf{LFS:} \url{https://www.linuxfromscratch.org/lfs/view/stable/}
    \item \textbf{LFS multilib fork:} \url{https://www.linuxfromscratch.org/~thomas/multilib/index.html}
    \item \textbf{BFLS:} \url{https://www.linuxfromscratch.org/blfs/downloads/stable/BLFS-BOOK-11.1-nochunks.html}
\end{itemize}

\section{Disks}
\begin{itemize}
    \item \textbf{FS comparison:} \url{https://en.wikipedia.org/wiki/Comparison_of_file_systems}
\end{itemize}


\end{document}
